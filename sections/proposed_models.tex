\begin{table*}[!ht]
  \small 
  \renewcommand{\arraystretch}{1.3}
  \setlength{\extrarowheight}{2pt}
  \caption{Comparison of Related Works on LoRa}
  \label{tab:related_works}
  \begin{tabularx}{\textwidth}{
      >{\raggedright\arraybackslash}p{0.15\textwidth}
      >{\raggedright\arraybackslash}X
      >{\raggedright\arraybackslash}p{0.18\textwidth}
      >{\raggedright\arraybackslash}p{0.18\textwidth}
      >{\raggedright\arraybackslash}p{0.20\textwidth}}
    \toprule
    \textbf{Referencia} & \textbf{Foco} & \textbf{Metodologia} 
      & \textbf{métricas} & \textbf{Lacuna} \\
    \midrule
      % colocar os outros artigos
      Lin, Huawei \emph{et al.} \cite{Lin2025UniGuardian} 
      & Proposta de defesa unificada (UniGuardian) contra ataques de LLM (injeção, backdoor, adversariais).
      & Propõe um framework de detecção unificado para analisar prompts e saídas.
      &  (1) auROC (Area Under the Receiver Operator Characteristic Curve) e (2) auPRC (Area Under the Precision-Recall Curve).
      & Falta de profundidade no que tange a ataques de backdoor mais complexos ou ofuscados, gerando falsos positivos. \\
    \midrule
    \textbf{This Work}
      & \textbf{}
      & \textbf{}
      & \textbf{}
      & \textbf{} \\
    \bottomrule
  \end{tabularx}
\end{table*}

\section{Proposed Model}
\label{sec:proposed_model}
